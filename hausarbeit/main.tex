%
%
\documentclass[11pt]{scrartcl}

% own geometry
%\usepackage[a4paper, left=3cm, right=3cm]{geometry}

\usepackage[ngerman]{babel} 
\usepackage[utf8]{inputenc} 
\usepackage[T1]{fontenc}
\usepackage{graphicx}
\usepackage{color}
\usepackage{xcolor}
\usepackage{jurabib}
\usepackage{hyperref}

\renewcommand*{\jbauthorfont}{\textsc}
\renewcommand*{\bibfnfont}{\normalfont}
\renewcommand*{\biblnfont}{\textsc}
%\renewcommand*{\samepageibidemname}{Ebd.}
\renewcommand*{\bibbtsep}{In: }
\renewcommand*{\bibjtsep}{In: }
\renewcommand*{\bibpldelim}{(}
\renewcommand*{\biburlprefix}{}
\renewcommand*{\biburlsuffix}{}

\makeatletter
\renewcommand*{\jbshorttitlefont}{%
\ifthenelse{%
\equal{\jb@@type}{article}%
\or
\equal{\jb@@type}{periodical}%
\or
\equal{\jb@@type}{incollection}%
}{%
\upshape%
}{%
\textit%
}%
}
\makeatother

\renewcommand*{\bibprdelim}{)}
\renewcommand*{\ajtsep}{}
\renewcommand*{\bpubaddr} { :}
\renewcommand*{\jbbtasep} { ; }
\renewcommand*{\jbbfsasep} { ; }
\renewcommand*{\jbbstasep} { ; }
\renewcommand*{\bibbtasep} { ; }
\renewcommand*{\bibbfsasep} { ; }
\renewcommand*{\bibbstasep} { ; } %between second and third author sep
\renewcommand*{\jbbtesep} { ; } %between two editors sep
\renewcommand*{\jbbfsesep} { ; } %between first and second editor sep
\renewcommand*{\jbbstesep} { ; } %between second and third editor sep
\renewcommand*{\bibbtesep} { ; } %between two editors sep
\renewcommand*{\bibbfsesep} { ; } %between first and second editor sep
\renewcommand*{\bibbstesep} { ; } %between second and third editor sep
\AddTo\bibsgerman{\def\editorsname{(Hrsg.)}}
\AddTo\bibsgerman{\def\editorname{(Hrsg.)}}
%\jurabibsetup{super, citefull=first,ibidem}
%\jurabibsetup{ibidem}
%\jurabibsetup{authorformat=citationreversed}
%\jurabibsetup{authorformat=reducedifibidem}
\jurabibsetup{biblikecite}
%\jurabibsetup{bibformat=ibidem}
%\jurabibsetup{pages=always}
\jbfirstcitepageranges
\AddTo\bibsgerman{\def\herename{hier}}
\jbuseidemhrule

\jurabibsetup{
  authorformat={smallcaps,year,and,citationreversed},
  titleformat={colonsep,all,italic},
  commabeforerest,
  see,
  dotafter=bibentry,
  ibidem=strict,
  biblikecite
}

\renewcommand*{\bibbtasep}{ und } %
\renewcommand*{\bibbfsasep}{, }   %
\renewcommand*{\bibbstasep}{ und }
\renewcommand*{\jbtitlefont}{}
\renewcommand*{\bibtfont}{}
\renewcommand*{\bibbtfont}{}
\renewcommand*{\bibjtfont}{}
\renewcommand*{\bibapifont}{}
\renewcommand*{\jbshorttitlefont}{}



	%
	% CITATIONS
	%
\newcommand{\book}[2]{\footnote{\cite[Vgl.][#2]{#1}}}
\newcommand{\bookwf}[2]{\cite[Vgl.][#2]{#1}}
\newcommand{\bookdir}[2]{\footnote{\cite[][#2]{#1}}}
\newcommand{\inetwf}[1]{\cite[Vgl.][\citefield{url}{#1}]{#1}}
\newcommand{\inetwfdir}[1]{\cite[][\citefield{url}{#1}]{#1}}
\newcommand{\inet}[1]{\footnote{\inetwf{#1}}}
\newcommand{\inetdir}[1]{\footnote{\cite[][\citefield{url}{#1}]{#1}}}
\newcommand{\innerref}[1]{\footnote{Vgl. auch Kapitel \ref{#1} dieser Arbeit, S. \pageref{#1}}}
\newcommand{\vgl}[2]{\cite[Vgl.][#2]{#1}}
\newcommand{\citeauthoryear}[1]{\citeauthor{#1} (\citeyear{#1})}
\bibliographystyle{jurabib}

% setup of source code listings
\usepackage{listings}
%\usepackage{courier}
\usepackage{caption}
\lstset{
	basicstyle=\footnotesize\ttfamily,	% default font
	numbers=left,						% line numbers placement
	numberstyle=\tiny,					% line numbers style
	%stepnumber=2,						% line number padding
	numbersep=5pt,						% padding between line numbers and code
	tabsize=2,							% 
	extendedchars=true,         
	breaklines=true,						% line breaks 
	keywordstyle=\color{red},
	frame=b,
	stringstyle=\color{gray}\ttfamily,	% color of strings in code
	showspaces=false,					% visualize spaces
    showtabs=false,						% visualize tabs
    xleftmargin=17pt,
	framexleftmargin=17pt,
	framexrightmargin=5pt,
	framexbottommargin=4pt,
	showstringspaces=false				% visualize spaces in strings        
 }
 
 \lstloadlanguages{% Check docs for further languages ...
         C,
         C++,
         bash
 }

\setlength{\parindent}{0pt}
\setlength{\parskip}\medskipamount

\DeclareCaptionFont{white}{\color{white}}
\DeclareCaptionFormat{listing}{\colorbox{gray}{\parbox{\textwidth}{#1#2#3}}}
\captionsetup[lstlisting]{format=listing,labelfont=white,textfont=white}

% layout the box
%\DeclareCaptionFormat{listing}{\colorbox[rgb]{0.43, 0.35, 0.35 {\parbox{\textwidth}{\hspace{15pt}#1#2#3}}}

% layout the caption ontop of code
\captionsetup[lstlisting]{format=listing,labelfont=white,textfont=white, singlelinecheck=false, margin=0pt, font={bf,footnotesize}}

% Headings
\usepackage{fancyhdr}
\fancyhead[R]{\colorbox{blue!20}{ Oliver Erxleben}}
\fancyfoot{}

% Document begins now
\begin{document}

\author{%
	Martin Helmich \small(\href{mailto:martin.helmich@hs-osnabrueck.de}{martin.helmich@hs-osnabrueck.de})\\%
	Oliver Erxleben \small(\href{mailto:oliver.erxleben@hs-osnabrueck.de}{oliver.erxleben@hs-osnabrueck.de})\\ \\%
	%
	Hochschule Osnabr"uck \\%
	Ingenieurswissenschaften und Informatik \\%
	Informatik - Mobile und Verteilte Anwendungen }

\title{\includegraphics[scale=0.75,keepaspectratio]{img/hs_os.png}\linebreak \linebreak Entwicklung eines Treiber und einer Toolchain zur Administration eines Embedded Systems}

\maketitle
\thispagestyle{empty}
\tableofcontents
\listoffigures

\lstlistoflistings
\thispagestyle{empty}
\pagebreak
\thispagestyle{empty}
\begin{abstract}
\textbf{Zusammenfassung:}\\ 	
Die vorliegende Arbeit wurde mit LaTeX verfasst und ist eine gemeinsame Arbeit von Oliver Erxleben und Martin Helmich für das Modul \textit{Hardwarenahe System- und Treiberprogrammierung} aus dem Master-Studiengang \textit{Informatik - Verteilte und Mobile Anwendungen} im Wintersemester 2012/13 an der Hochschule Osnabrück / University of Applied Sciences. \\ 
\\
Das Thema der Arbeit lautet \textit{Entwicklung eines Treiber und einer Toolchain zur Administration eines Embedded Systems}. Im Kern beschreibt die Arbeit die Entwicklung eines Treibers für ein Embbed Systems. Genauer wird zum einen ein Linux-Treiber für eine ZPU vorgestellt und zum anderen Bibliotheken und Wekrzeuge für die einfache Verwendung in Anwenderprogrammen.\\
Im ersten Teil der Arbeit wird das Konzept der Hardware und das Konzept des Treiber und Der Programmierbibliotheken vorgestellt. In den darauffolgenden Abschnitten werden detailliert die einzelnen Komponenten beschrieben. Anwenderprogramme zum Testen werden aus an geeigneten Stellen vorgestellt. Details über die Implementierung wird aus Gründen der Lesbarkeit in den Anhang verschoben.\\ 
\\
Sofern nicht anders angegeben, wird zum Kompilieren des Quellcodes der GCC verwendet. 
\end{abstract}

\pagebreak
% set new page style

\pagestyle{fancy}
\setcounter{page}{1} 

\section{Architektur}

\section{Linux-Treiber-Modul: modzpu}

\section{ZPU Lib: libzpu}

\section{Intel Hex Lib: libcintelhex}

\pagebreak % content ends here

\fancyhead[R]{}

\thispagestyle{empty}

\renewcommand*{\biburlprefix}{(URL: }
\renewcommand*{\biburlsuffix}{)}

\pagebreak
\addcontentsline{toc}{section}{Literaturverzeichnis} % Eintrag ins Inhaltsverzeichnis
\bibliography{bib/bib}

\appendix

\end{document}
