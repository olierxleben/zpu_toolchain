%
%
\documentclass[11pt]{scrartcl}

% own geometry
%\usepackage[a4paper, left=3cm, right=3cm]{geometry}

\usepackage[ngerman]{babel} 
\usepackage[utf8]{inputenc} 
\usepackage[T1]{fontenc}
\usepackage{graphicx}
\usepackage{color}
\usepackage{xcolor}
\usepackage{jurabib}
\usepackage{hyperref}

\renewcommand*{\jbauthorfont}{\textsc}
\renewcommand*{\bibfnfont}{\normalfont}
\renewcommand*{\biblnfont}{\textsc}
%\renewcommand*{\samepageibidemname}{Ebd.}
\renewcommand*{\bibbtsep}{In: }
\renewcommand*{\bibjtsep}{In: }
\renewcommand*{\bibpldelim}{(}
\renewcommand*{\biburlprefix}{}
\renewcommand*{\biburlsuffix}{}

\makeatletter
\renewcommand*{\jbshorttitlefont}{%
\ifthenelse{%
\equal{\jb@@type}{article}%
\or
\equal{\jb@@type}{periodical}%
\or
\equal{\jb@@type}{incollection}%
}{%
\upshape%
}{%
\textit%
}%
}
\makeatother

\renewcommand*{\bibprdelim}{)}
\renewcommand*{\ajtsep}{}
\renewcommand*{\bpubaddr} { :}
\renewcommand*{\jbbtasep} { ; }
\renewcommand*{\jbbfsasep} { ; }
\renewcommand*{\jbbstasep} { ; }
\renewcommand*{\bibbtasep} { ; }
\renewcommand*{\bibbfsasep} { ; }
\renewcommand*{\bibbstasep} { ; } %between second and third author sep
\renewcommand*{\jbbtesep} { ; } %between two editors sep
\renewcommand*{\jbbfsesep} { ; } %between first and second editor sep
\renewcommand*{\jbbstesep} { ; } %between second and third editor sep
\renewcommand*{\bibbtesep} { ; } %between two editors sep
\renewcommand*{\bibbfsesep} { ; } %between first and second editor sep
\renewcommand*{\bibbstesep} { ; } %between second and third editor sep
\AddTo\bibsgerman{\def\editorsname{(Hrsg.)}}
\AddTo\bibsgerman{\def\editorname{(Hrsg.)}}
%\jurabibsetup{super, citefull=first,ibidem}
%\jurabibsetup{ibidem}
%\jurabibsetup{authorformat=citationreversed}
%\jurabibsetup{authorformat=reducedifibidem}
\jurabibsetup{biblikecite}
%\jurabibsetup{bibformat=ibidem}
%\jurabibsetup{pages=always}
\jbfirstcitepageranges
\AddTo\bibsgerman{\def\herename{hier}}
\jbuseidemhrule

\jurabibsetup{
  authorformat={smallcaps,year,and,citationreversed},
  titleformat={colonsep,all,italic},
  commabeforerest,
  see,
  dotafter=bibentry,
  ibidem=strict,
  biblikecite
}

\renewcommand*{\bibbtasep}{ und } %
\renewcommand*{\bibbfsasep}{, }   %
\renewcommand*{\bibbstasep}{ und }
\renewcommand*{\jbtitlefont}{}
\renewcommand*{\bibtfont}{}
\renewcommand*{\bibbtfont}{}
\renewcommand*{\bibjtfont}{}
\renewcommand*{\bibapifont}{}
\renewcommand*{\jbshorttitlefont}{}



	%
	% CITATIONS
	%
\newcommand{\book}[2]{\footnote{\cite[Vgl.][#2]{#1}}}
\newcommand{\bookwf}[2]{\cite[Vgl.][#2]{#1}}
\newcommand{\bookdir}[2]{\footnote{\cite[][#2]{#1}}}
\newcommand{\inetwf}[1]{\cite[Vgl.][\citefield{url}{#1}]{#1}}
\newcommand{\inetwfdir}[1]{\cite[][\citefield{url}{#1}]{#1}}
\newcommand{\inet}[1]{\footnote{\inetwf{#1}}}
\newcommand{\inetdir}[1]{\footnote{\cite[][\citefield{url}{#1}]{#1}}}
\newcommand{\innerref}[1]{\footnote{Vgl. auch Kapitel \ref{#1} dieser Arbeit, S. \pageref{#1}}}
\newcommand{\vgl}[2]{\cite[Vgl.][#2]{#1}}
\newcommand{\citeauthoryear}[1]{\citeauthor{#1} (\citeyear{#1})}
\bibliographystyle{jurabib}

% setup of source code listings
\usepackage{listings}
%\usepackage{courier}
\usepackage{caption}
\lstset{
	basicstyle=\footnotesize\ttfamily,	% default font
	numbers=left,						% line numbers placement
	numberstyle=\tiny,					% line numbers style
	%stepnumber=2,						% line number padding
	numbersep=5pt,						% padding between line numbers and code
	tabsize=2,							% 
	extendedchars=true,         
	breaklines=true,						% line breaks 
	keywordstyle=\color{red},
	frame=b,
	stringstyle=\color{gray}\ttfamily,	% color of strings in code
	showspaces=false,					% visualize spaces
    showtabs=false,						% visualize tabs
    xleftmargin=17pt,
	framexleftmargin=17pt,
	framexrightmargin=5pt,
	framexbottommargin=4pt,
	showstringspaces=false				% visualize spaces in strings        
 }
 
 \lstloadlanguages{% Check docs for further languages ...
         C,
         C++,
         bash
 }

\setlength{\parindent}{0pt}
\setlength{\parskip}\medskipamount

\DeclareCaptionFont{white}{\color{white}}
\DeclareCaptionFormat{listing}{\colorbox{gray}{\parbox{\textwidth}{#1#2#3}}}
\captionsetup[lstlisting]{format=listing,labelfont=white,textfont=white}

% layout the box
%\DeclareCaptionFormat{listing}{\colorbox[rgb]{0.43, 0.35, 0.35 {\parbox{\textwidth}{\hspace{15pt}#1#2#3}}}

% layout the caption ontop of code
\captionsetup[lstlisting]{format=listing,labelfont=white,textfont=white, singlelinecheck=false, margin=0pt, font={bf,footnotesize}}

% Headings
\usepackage{fancyhdr}
\fancyhead[R]{\colorbox{blue!20}{ Oliver Erxleben}}
\fancyfoot{}

% Document begins now
\begin{document}

\author{%
	Martin Helmich \small(\href{mailto:martin.helmich@hs-osnabrueck.de}{martin.helmich@hs-osnabrueck.de})\\%
	Oliver Erxleben \small(\href{mailto:oliver.erxleben@hs-osnabrueck.de}{oliver.erxleben@hs-osnabrueck.de})\\ \\%
	%
	Hochschule Osnabr"uck \\%
	Ingenieurswissenschaften und Informatik \\%
	Informatik - Mobile und Verteilte Anwendungen }

\title{\includegraphics[scale=0.75,keepaspectratio]{img/hs_os.png}\linebreak \linebreak Entwicklung eines Treiber und einer Toolchain zur Administration eines Embedded Systems}

\maketitle
\thispagestyle{empty}
\tableofcontents
\listoffigures

\lstlistoflistings
\thispagestyle{empty}
\pagebreak
\thispagestyle{empty}
\begin{abstract}
\textbf{Zusammenfassung:}\\ 	
Die vorliegende Arbeit wurde mit LaTeX verfasst und ist eine gemeinsame Arbeit von Oliver Erxleben und Martin Helmich für das Modul \textit{Parallele und verteilte Algorithmen} aus dem Master-Studiengang \textit{Informatik - Verteilte und Mobile Anwendungen} im Wintersemester 2012/13 an der Hochschule Osnabrück / University of Applied Sciences. \\ 
\\
Die Arbeit gliedert sich in zwei Teile. Im ersten Teil wird eine Einführung in OpenMP gegeben. Es werden hier Kernfunktionen der Programmierschnittstelle gezeigt, die Parallelisierung von Schleifen verdeutlicht und auf Verteilung von Aufgaben eingegangen.\\
Der zweite Teil zeigt die Optimierung der Parallelisierungsaufgabe aus Intel`s \textit{Accelerate Your Code} von September 2012 mittels OpenMP auf und vergleicht die Ergebnisse mit der Programmierbibliothek \textit{Threading Building Blocks} von Intel.\\
\\
Code-Beispiele werden aus Gründen der Lesbarkeit und des Umfangs in die Anhänge verschoben. An den Stellen wird auf die Source Codes verwiesen. Die Arbeit betrachtet weiterhin nur OpenMP in Verwendung mit C und C++. Fortran wird nicht behandelt.
\end{abstract}

\pagebreak
% set new page style

\pagestyle{fancy}
\setcounter{page}{1} 

\pagebreak % content ends here

\fancyhead[R]{}

\thispagestyle{empty}

\renewcommand*{\biburlprefix}{(URL: }
\renewcommand*{\biburlsuffix}{)}

\pagebreak
\addcontentsline{toc}{section}{Literaturverzeichnis} % Eintrag ins Inhaltsverzeichnis
\bibliography{bib/bib}

\appendix

\end{document}
