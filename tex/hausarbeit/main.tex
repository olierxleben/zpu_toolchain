%
%
\documentclass[12pt]{scrartcl}

% own geometry
%\usepackage[a4paper, left=3cm, right=3cm]{geometry}

\usepackage[ngerman]{babel} 
\usepackage[utf8]{inputenc} 
\usepackage[T1]{fontenc}
\usepackage{graphicx}
\usepackage{color}
\usepackage{xcolor}
\usepackage{jurabib}
\usepackage{hyperref}
\usepackage{tikz}
\renewcommand*{\jbauthorfont}{\textsc}
\renewcommand*{\bibfnfont}{\normalfont}
\renewcommand*{\biblnfont}{\textsc}
%\renewcommand*{\samepageibidemname}{Ebd.}
\renewcommand*{\bibbtsep}{In: }
\renewcommand*{\bibjtsep}{In: }
\renewcommand*{\bibpldelim}{(}
\renewcommand*{\biburlprefix}{}
\renewcommand*{\biburlsuffix}{}

\makeatletter
\renewcommand*{\jbshorttitlefont}{%
\ifthenelse{%
\equal{\jb@@type}{article}%
\or
\equal{\jb@@type}{periodical}%
\or
\equal{\jb@@type}{incollection}%
}{%
\upshape%
}{%
\textit%
}%
}
\makeatother

\renewcommand*{\bibprdelim}{)}
\renewcommand*{\ajtsep}{}
\renewcommand*{\bpubaddr} { :}
\renewcommand*{\jbbtasep} { ; }
\renewcommand*{\jbbfsasep} { ; }
\renewcommand*{\jbbstasep} { ; }
\renewcommand*{\bibbtasep} { ; }
\renewcommand*{\bibbfsasep} { ; }
\renewcommand*{\bibbstasep} { ; } %between second and third author sep
\renewcommand*{\jbbtesep} { ; } %between two editors sep
\renewcommand*{\jbbfsesep} { ; } %between first and second editor sep
\renewcommand*{\jbbstesep} { ; } %between second and third editor sep
\renewcommand*{\bibbtesep} { ; } %between two editors sep
\renewcommand*{\bibbfsesep} { ; } %between first and second editor sep
\renewcommand*{\bibbstesep} { ; } %between second and third editor sep
\AddTo\bibsgerman{\def\editorsname{(Hrsg.)}}
\AddTo\bibsgerman{\def\editorname{(Hrsg.)}}
%\jurabibsetup{super, citefull=first,ibidem}
%\jurabibsetup{ibidem}
%\jurabibsetup{authorformat=citationreversed}
%\jurabibsetup{authorformat=reducedifibidem}
\jurabibsetup{biblikecite}
%\jurabibsetup{bibformat=ibidem}
%\jurabibsetup{pages=always}
\jbfirstcitepageranges
\AddTo\bibsgerman{\def\herename{hier}}
\jbuseidemhrule

\jurabibsetup{
  authorformat={smallcaps,year,and,citationreversed},
  titleformat={colonsep,all,italic},
  commabeforerest,
  see,
  dotafter=bibentry,
  ibidem=strict,
  biblikecite
}

\renewcommand*{\bibbtasep}{ und } %
\renewcommand*{\bibbfsasep}{, }   %
\renewcommand*{\bibbstasep}{ und }
\renewcommand*{\jbtitlefont}{}
\renewcommand*{\bibtfont}{}
\renewcommand*{\bibbtfont}{}
\renewcommand*{\bibjtfont}{}
\renewcommand*{\bibapifont}{}
\renewcommand*{\jbshorttitlefont}{}



	%
	% CITATIONS
	%
\newcommand{\book}[2]{\footnote{\cite[Vgl.][#2]{#1}}}
\newcommand{\bookwf}[2]{\cite[Vgl.][#2]{#1}}
\newcommand{\bookdir}[2]{\footnote{\cite[][#2]{#1}}}
\newcommand{\inetwf}[1]{\cite[Vgl.][\citefield{url}{#1}]{#1}}
\newcommand{\inetwfdir}[1]{\cite[][\citefield{url}{#1}]{#1}}
\newcommand{\inet}[1]{\footnote{\inetwf{#1}}}
\newcommand{\inetdir}[1]{\footnote{\cite[][\citefield{url}{#1}]{#1}}}
\newcommand{\innerref}[1]{\footnote{Vgl. auch Kapitel \ref{#1} dieser Arbeit, S. \pageref{#1}}}
\newcommand{\vgl}[2]{\cite[Vgl.][#2]{#1}}
\newcommand{\citeauthoryear}[1]{\citeauthor{#1} (\citeyear{#1})}
\bibliographystyle{jurabib}

% setup of source code listings
\usepackage{listings}
%\usepackage{courier}
\usepackage{caption}
\lstset{
	basicstyle=\footnotesize\ttfamily,	% default font
	numbers=left,						% line numbers placement
	numberstyle=\tiny,					% line numbers style
	%stepnumber=2,						% line number padding
	numbersep=5pt,						% padding between line numbers and code
	tabsize=2,							% 
	extendedchars=true,         
	breaklines=true,						% line breaks 
	keywordstyle=\color{red},
	frame=lrtb,
	stringstyle=\color{gray}\ttfamily,	% color of strings in code
	showspaces=false,					% visualize spaces
    showtabs=false,						% visualize tabs
    xleftmargin=17pt,
	framexleftmargin=17pt,
	framexrightmargin=5pt,
	framexbottommargin=4pt,
	showstringspaces=false				% visualize spaces in strings        
 }
 
 \lstloadlanguages{% Check docs for further languages ...
         C,
         C++,
         bash
 }

\setlength{\parindent}{0pt}
\setlength{\parskip}\medskipamount

\tikzstyle{every node}=[font=\footnotesize]

%\DeclareCaptionFont{white}{\color{white}}
%\DeclareCaptionFormat{listing}{\colorbox{gray}{\parbox{\textwidth}{#1#2#3}}}
%\captionsetup[lstlisting]{format=listing,labelfont=white,textfont=white}

% layout the box
%\DeclareCaptionFormat{listing}{\colorbox[rgb]{0.43, 0.35, 0.35 {\parbox{\textwidth}{\hspace{15pt}#1#2#3}}}

% layout the caption ontop of code
%\captionsetup[lstlisting]{format=listing,labelfont=white,textfont=white, singlelinecheck=false, margin=0pt, font={bf,footnotesize}}

% Headings
\usepackage{fancyhdr}
\fancyhead[R]{\colorbox{blue!20}{ Oliver Erxleben}}
\fancyfoot{}

% Document begins now
\begin{document}

\author{%
	Martin Helmich \small(\href{mailto:martin.helmich@hs-osnabrueck.de}{martin.helmich@hs-osnabrueck.de})\\%
	Oliver Erxleben \small(\href{mailto:oliver.erxleben@hs-osnabrueck.de}{oliver.erxleben@hs-osnabrueck.de})\\ \\%
	%
	Hochschule Osnabr"uck \\%
	Ingenieurswissenschaften und Informatik \\%
	Informatik - Mobile und Verteilte Anwendungen }

\title{\includegraphics[scale=0.75,keepaspectratio]{img/hs_os.png}\linebreak \linebreak Entwicklung eines Treiber und einer Toolchain zur Administration eines Embedded Systems}

\maketitle
\thispagestyle{empty}
\tableofcontents

\listoffigures

\lstlistoflistings

\thispagestyle{empty}
\pagebreak
\thispagestyle{empty}
\begin{abstract}
\textbf{Zusammenfassung:}\\ 	
Die vorliegende Arbeit wurde mit LaTeX verfasst und ist eine gemeinsame Arbeit von Oliver Erxleben und Martin Helmich für das Modul \textit{Hardwarenahe System- und Treiberprogrammierung} aus dem Master-Studiengang \textit{Informatik - Verteilte und Mobile Anwendungen} im Wintersemester 2012/13 an der Hochschule Osnabrück / University of Applied Sciences. \\ 
\\
Das Thema der Arbeit lautet \textit{Entwicklung eines Treiber und einer Toolchain zur Administration eines Embedded Systems}. Im Kern beschreibt die Arbeit die Entwicklung eines Treibers für ein eingebettetes System unter Linux. Treiberprogrammierung unter anderen Systemen, wie zum Beispiel Microsoft Windows oder Apple Mac OS werden nicht betrachtet. Genauer wird zum einen ein Linux-Treiber für eine ZPU vorgestellt und zum anderen Bibliotheken und Wekrzeuge für die einfache Verwendung in Anwenderprogrammen zur Steuerung des ZPU entwickelt.\\
Im ersten Teil der Arbeit wird das Konzept der Hardware und das Konzept des Treibers und der Programmierbibliotheken vorgestellt. In den darauffolgenden Abschnitten werden detailliert die einzelnen Komponenten beschrieben. Anwenderprogramme zum Testen werden an geeigneten Stellen vorgestellt. \\ 

Details über die Implementierung der Anwenderprogrammen wird aus Gründen der Lesbarkeit in den Anhang verschoben.\\ 
\\
% TODO Kernel Version ???
Sofern nicht anders angegeben, wird zum Kompilieren des Quellcodes der GNU-Compiler GCC unter Linux verwendet. 
\end{abstract}

\pagebreak
% set new page style

\pagestyle{fancy}
\setcounter{page}{1} 

\section{Einleitung}

% TODO: überarbeiten / erweitern
Täglich verwendet und verlässt sich unsere Gesellschaft auf computergestützte Anwendungen. Wir suchen, schreiben, drucken, kopieren, erstellen oder entfernen Daten. Nur allzu oft werden dabei Geräte verwendet, die Funktionen für den Anwender bereitstellen, oder aber das Funktionsspektrum des Comutersystems erweitern. Funktionen der Computerkomponenten werden durch Anwenderprogramme durchgeführt, die von einem Betriebssystem verwaltet werden. Das Betriebssystem verwaltet auch alle Komponenten des Computersystems. Diese Komponenten sind Software oder auch Hardware. Erst die Teamarbeit zwischen Hardware und Software ermöglicht für den Anwender die Ausführung komplexer Programme. \\
Das Betriebssystem steuert den Zugriff auf Hardware und benötigt Kenntnisse über die angeschlossenen Hardware-Komonenten. Das \textit{Wissen} über angeschlossene Hardwarekomponenten ist in sog. Gerätetreibern\footnote{Gerätetreiber:} hinterlegt und stellt einen Teil des Betriebssystemkerns dar, der für den Zugriff auf eine Hardware verantwortlich ist. Für jedes Gerät wird ein eigener Treiber benötigt.\\
Die Entwicklung eines Gerätetribers und zugehörige Softwarekomponenten stellen einen interessanten, wenn nicht sogar erstrebenswerten, Zweig der Programmierung von Computersystemen dar. \\
\\


\section{Architektur}
% TODO: verfassen

\subsection{ZPU-}

\begin{figure}[!htb]
	\begin{center}
		%include from external
		\begin{tikzpicture}
    \tikzstyle{every node}=[font=\small]
	\usetikzlibrary{positioning}
	\tikzstyle{pci}=[draw,fill=green!50,rectangle]
	\tikzstyle{krn}=[draw,fill=yellow!50,rectangle]
				
	\draw [fill=green!20] (0,-0.5) rectangle +(8,2.5);
	\draw [fill=yellow!20] (0,2.5) rectangle +(8,2.5);
				
	%\draw [<->] (1.8,0.9) -- (5,0.9);
	%\draw [<->] (3.8,1.1) -- (5,1.1);
	\draw [<->,dotted] (6,1.5) -- (6,3);
			
	\draw [<->,dotted] (1,0.2) -- (1,-0.2) -- (6.1,-0.2) -- (6.1,0.5);
	\draw [<->,dotted] (3,0.2) -- (3,-0) -- (5.9,0) -- (5.9,0.5);
				
	\draw [<->,dotted] (3.2,4.3) -- (3.2,4.5) -- (5.9,4.5) -- (5.9,4);
	\draw [<->,dotted] (1.2,4.3) -- (1.2,4.7) -- (6.1,4.7) -- (6.1,4);
				
	%\draw [<->] (1.8,3.4) -- (5,3.4);
	%\draw [<->] (3.8,3.6) -- (5,3.6);
				
	\node [anchor=north east] at (8,-0.5) {PCI-Hardware};
	\node [anchor=south east] at (8, 5.0) {Kernel};
				
	\draw [pci] (0.2,0.2) rectangle +(1.6,1.6) node [align=left,midway] {Eingabe-\\FIFO\\(\texttt{stdin})};
	\draw [pci] (2.2,0.2) rectangle +(1.6,1.6) node [midway,align=left] {Ausgabe-\\FIFO\\(\texttt{stdout})};
					
	\draw [krn] (0.2,2.7) rectangle +(1.6,1.6) node [midway,align=left] {Eingabe-\\FIFO};
	\draw [krn] (2.2,2.7) rectangle +(1.6,1.6) node [midway,align=left] {Ausgabe-\\FIFO};
				
	\draw [->] (1,2.7) -- (1,1.8);
	\draw [<-] (3,2.7) -- (3,1.8);
				
	\draw [->] (1,5.5) -- (1,4.3);
	\draw [<-] (3,5.5) -- (3,4.3);
	\node [anchor=south] at (3,5.5) {\texttt{read()}};
	\node [anchor=south] at (1,5.5) {\texttt{write()}};
				
	\draw [pci] (5,0.5) rectangle +(2,1) node [align=left,midway] {Interrupt-\\Controller};
		\draw [krn] (5,3) rectangle +(2,1) node [midway] {IRQ-Handler};
\end{tikzpicture}
		\caption{ZPU-Ein und Ausgabe}
		\label{zpu_io}
	\end{center}
\end{figure}


\section{Linux-Treiber-Modul: modzpu}

\begin{figure}[!htb]
	\begin{center}
		\begin{center}
\begin{tikzpicture}
	\usetikzlibrary{positioning}
	\tikzstyle{node}=[anchor=mid,text centered,minimum height=2em];
	\tikzstyle{app}=[draw,fill=green!50,rectangle]
	\tikzstyle{lib}=[draw,fill=yellow!50,rectangle]
	\tikzstyle{dri}=[draw,fill=red!50,rectangle]
				
	\draw [lib] (6,0.5) rectangle +(2,0.75) node [node,midway] {libzpu};
	\draw [lib] (4cm-2pt,0.5) rectangle +(2,0.75) node [node,midway] {libintelhex};
				
	\draw [dri] (-0.5,-0.5) rectangle +(8.5,-0.75) node [node,midway] {modzpu};
	\draw [app] (6,2) rectangle +(2,0.75) node [node,midway] {zpuio};
	\draw [app] (4cm-2pt,2) rectangle +(2,0.75) node [node,midway] {zpuload};
					
	\draw [dashed] (0,0) -- (8,0);
	\draw [->] (7,0.5) -- (7,-0.5);
	\draw [->] (7,2) -- (7,1.25);
	\draw [->] (5,2) -- (5,1.25);
	\draw [->] (5.5,2) -- (6.5,1.25);
				
	\node [anchor=west] at (-0.5, 2.375) {Userspace-Programme};
	\node [anchor=west] at (-0.5, 0.875) {Bibliotheken};
\end{tikzpicture}
\end{center}
		\caption{Überblick ZPU-Toolchain}
		\label{zpu_architecture}	
	\end{center}
\end{figure}

\section{ZPU Lib: libzpu}
% TODO: verfassen
\section{Parsing der Intel HEX-Dateien: Die libcintelhex}

Die Logik zum Parsen der Intel HEX-formatierten Eingabedateien wurde in eine eigene Programmbibliothek ausgegliedert.

\subsection{Installation}

Die Installation der cintelhex-Bibliothek erfolgt mit den Unix-üblichen \texttt{configure}- und \texttt{make}-Befehlen:

\begin{lstlisting}[caption=Installation der cintelhex-Bibliothek]
> ./configure # optional: --prefix=/usr/local
> make
> make install
> ldconfig
\end{lstlisting}

Bei der Installation wird die Datei \texttt{libcintelhex.so} nach \texttt{<prefix>/lib}, und die Datei \texttt{cintelhex.h} nach \texttt{<prefix>/include} installiert. Das Standardpräfix ist \texttt{/usr/local}.

Bei Verwendung des Standardpräfix kann die cintelhex-Bibliothek beim Linken durch den GCC-Parameter \texttt{-l cintelhex} aktiviert werden.

\subsection{Schnittstellenbeschreibung}

Die Methoden der cintelhex-Bibliothek können durch inkludieren der Header-Datei \texttt{cintelhex.h} verwendet werden. In der Datei werden folgende wichtigen Methoden deklariert:

\begin{description}
	\item[ihex\_recordset\_t* ihex\_rs\_from\_file(char* filename)] Diese Methode liest den Inhalt einer Intel HEX-Datei (der Dateiname wird durch \texttt{filename} beschrieben) in eine \texttt{ihex\_recordset\_t}-Struktur (siehe auch Abschnitt \ref{ihex_structs}). Sie liefert einen Zeiger auf die ausgelesene Struktur zurück, oder \texttt{NULL} im Fehlerfall.
	\item[ihex\_recordset\_t* ihex\_rs\_from\_string(char* data)] Diese Methode verhält sich genau wie \texttt{ihex\_rs\_from\_file}, erwartet als Eingabeparameter jedoch direkt einen Zeiger auf einen Intel HEX-String.
	\item[int ihex\_mem\_copy(ihex\_recordset\_t *rs, void* dst, ulong\_t n)] Diese Methode kopiert das durch eine Intel HEX-Datei beschriebene Programm an eine beliebige Stelle im Speicher. Das zu ladende Programm wird durch die Struktur \texttt{rs} beschrieben. Der Zeiger \texttt{dst} beschreibt die Zielposition im Speicher (dies kann auch ein durch \texttt{mmap} abgebildeter Gerätespeicher sein). \texttt{n} beschreibt die Größe des Zielbereichs. Die Methode liefert im Erfolgsfall 0, ansonsten einen Fehlercode zurück.
	\item[char* ihex\_error()] Diese Methode liefert eine Beschreibung des zuletzt aufgetretenen Fehlers zurück. Falls kein Fehler aufgetreten ist, wird \texttt{NULL} zurückgeliefert.
	\item[ihex\_error\_t ihex\_errno()] Diese Methode liefert den Fehlercode des zuletzt aufgetretenen Fehlers zurück. Falls kein Fehler aufgetreten ist, wird \texttt{0} zurückgeliefert.
\end{description}

\subsection{Datenstrukturen}
\label{ihex_structs}

Die cintelhex-Bibliothek definiert die Datenstrukturen \texttt{ihex\_recordset\_t} und \texttt{ihex\_record\_t}. Sie sind wie folgt definiert:

\begin{lstlisting}[language=C,morekeywords={uint\_t, ihex\_rtype\_t, ihex\_addr\_t, ihex\_rdata\_t, ihex\_rchks\_t, ihex\_record\_t, ihex\_recordset\_t}, caption=Definition der \texttt{ihex\_record\_t}- und \texttt{ihex\_recordset\_t}-Strukturen]
/// Models a single Intel HEX record.
typedef struct ihex_record {
	uint_t       ihr_length;   //!< Length of the record in bytes.
	ihex_rtype_t ihr_type;     //!< Record type (see ihex_rtype_t).
	ihex_addr_t  ihr_address;  //!< Base address offset.
	ihex_rdata_t ihr_data;     //!< Record data.
	ihex_rchks_t ihr_checksum; //!< The record's checksum.
} ihex_record_t;

/// Models a set of Intel HEX records.
typedef struct ihex_recordset {
	uint_t         ihrs_count;   //!< Amount of records.
	ihex_record_t *ihrs_records; //!< A list of record (with ihrs_count elements).
} ihex_recordset_t;
\end{lstlisting}

Abbildung \ref{lcih_datastructs} zeigt zudem die Beziehungen der Datenstrukturen untereinander. Das Attribut \texttt{ihex\_data} der Struktur \texttt{ihex\_record\_t} ist ein Zeiger auf ein \texttt{uint8\_t}-Feld der Länge \texttt{ihr\_length}.

\begin{figure}[!htb]
	%include from externl
	\begin{center}
		\begin{tikzpicture}

	\usetikzlibrary{arrows}
	\tikzstyle{attr}=[midway,text width=3.3cm,text height=0.3cm]
	\tikzstyle{green}=[fill=green!20]
	\tikzstyle{yellow}=[fill=yellow!20]	
			
	\draw [fill=green!50] (0,0) rectangle +(3.5,0.5) node [attr] {\texttt{ihex\_recordset\_t}};
	\draw [green] (0,-0.5) rectangle +(3.5,0.5) node [attr] {\texttt{ihrs\_count} = 2};
	\draw [green] (0,-1) rectangle +(3.5,0.5) node [attr] {\texttt{ihrs\_records} = };
			
	\draw [fill=yellow!50] (0,-2) rectangle +(3.5,0.5) node [attr] {\texttt{ihex\_record\_t}};
	\draw [yellow] (0,-2.5) rectangle +(3.5,0.5) node [attr] {\texttt{ihr\_length} = 4};
	\draw [yellow] (0,-3) rectangle +(3.5,0.5) node [attr] {\texttt{ihr\_type} = 0x00};
	\draw [yellow] (0,-3.5) rectangle +(3.5,0.5) node [attr] {\texttt{ihr\_addr} = 0x00};
	\draw [yellow] (0,-4) rectangle +(3.5,0.5) node [attr] {\texttt{ihr\_data} = };
			
	\draw [fill=yellow!50] (3.5,-2) rectangle +(3.5,0.5) node [attr] {\texttt{ihex\_record\_t}};
	\draw [yellow] (3.5,-2.5) rectangle +(3.5,0.5) node [attr] {\texttt{ihr\_length} = 6};
	\draw [yellow] (3.5,-3) rectangle +(3.5,0.5) node [attr] {\texttt{ihr\_type} = 0x00};
	\draw [yellow] (3.5,-3.5) rectangle +(3.5,0.5) node [attr] {\texttt{ihr\_addr} = 0x04};
	\draw [yellow] (3.5,-4) rectangle +(3.5,0.5) node [attr] {\texttt{ihr\_data} = };
			
	\draw[*->] (3,-0.7) -- (3,-1.5);
	\draw[*->] (2.5,-3.7) -- (2.5,-4.5) -- (0.75/2, -4.5) -- (0.75/2,-5);
	\draw[*->] (6,-3.7) -- (6,-4.5) -- (3.5+0.75/2, -4.5) -- (3.5+0.75/2,-5);
			
	\foreach \x in {0,0.75,1.5,2.25}{
		\draw (\x, -5.5) rectangle +(0.75,0.5) node [attr,text width=0.7cm] {0x00};
	};
			
	\foreach \x in {3.5,4.25,5,5.75,6.5,7.25}{
		\draw (\x, -5.5) rectangle +(0.75,0.5) node [attr,text width=0.7cm] {0x00};
	};
\end{tikzpicture}
	\end{center}
	\caption{Datenstrukturen der libcintelhex}
	\label{lcih_datastructs}
\end{figure}

\subsection{Fehlercodes}

Die \texttt{cintelhex.h} definiert die folgenden Fehlerkonstanten:

\begin{description}
	\item[IHEX\_ERR\_INCORRECT\_CHECKSUM] Tritt auf, wenn die Prüfsumme eines Eintrages ungültig ist. Zur Verifizierung der Prüfsumme werden alle Bytes eines Eintrages (einschließlich der Prüfsumme selbst) aufsummiert. Ist das unterste Byte der Summe $=0$, ist die Prüfsumme korrekt.
	\item[IHEX\_ERR\_NO\_EOF] Tritt auf, wenn die Eingabedatei keinen EOF-Eintrag (Typ \texttt{0x01}) enthält.
	\item[IHEX\_ERR\_PARSE\_ERROR]
	\item[IHEX\_ERR\_WRONG\_RECORD\_LENGTH]
	\item[IHEX\_ERR\_NO\_INPUT]
	\item[IHEX\_ERR\_UNKNOWN\_RECORD\_TYPE]
	\item[IHEX\_ERR\_PREMATURE\_EOF]
	\item[IHEX\_ERR\_ADDRESS\_OUT\_OF\_RANGE]
	\item[IHEX\_ERR\_MMAP\_FAILED]
\end{description}

\subsection{Anwendungsbeispiel}

\begin{lstlisting}[language=C,caption=Anwendungsbeispiel der cintelhex-Bibliothek]
#include <stdlib.h>
#include <cintelhex.h>

int main()
{
	ihex_recordset_t* r = ihex_rs_from_file("input.hex");
	void*             d = malloc(8192);
	
	if (r != NULL)
	{
		return ihex_mem_copy(r, d, 8192);
	}
	return ihex_errno();
}
\end{lstlisting}

\pagebreak % content ends here

\fancyhead[R]{}

\thispagestyle{empty}

\renewcommand*{\biburlprefix}{(URL: }
\renewcommand*{\biburlsuffix}{)}

\pagebreak
\addcontentsline{toc}{section}{Literaturverzeichnis} % Eintrag ins Inhaltsverzeichnis
\bibliography{bib/bib}

\appendix

\end{document}
