%
%
\documentclass[12pt]{scrartcl}

% own geometry
%\usepackage[a4paper, left=3cm, right=3cm]{geometry}

\usepackage[ngerman]{babel} 
\usepackage[utf8]{inputenc} 
\usepackage[T1]{fontenc}
\usepackage{graphicx}
\usepackage{color}
\usepackage{xcolor}
\usepackage{jurabib}
\usepackage{hyperref}
\usepackage{tikz}
\renewcommand*{\jbauthorfont}{\textsc}
\renewcommand*{\bibfnfont}{\normalfont}
\renewcommand*{\biblnfont}{\textsc}
%\renewcommand*{\samepageibidemname}{Ebd.}
\renewcommand*{\bibbtsep}{In: }
\renewcommand*{\bibjtsep}{In: }
\renewcommand*{\bibpldelim}{(}
\renewcommand*{\biburlprefix}{}
\renewcommand*{\biburlsuffix}{}

\makeatletter
\renewcommand*{\jbshorttitlefont}{%
\ifthenelse{%
\equal{\jb@@type}{article}%
\or
\equal{\jb@@type}{periodical}%
\or
\equal{\jb@@type}{incollection}%
}{%
\upshape%
}{%
\textit%
}%
}
\makeatother

\renewcommand*{\bibprdelim}{)}
\renewcommand*{\ajtsep}{}
\renewcommand*{\bpubaddr} { :}
\renewcommand*{\jbbtasep} { ; }
\renewcommand*{\jbbfsasep} { ; }
\renewcommand*{\jbbstasep} { ; }
\renewcommand*{\bibbtasep} { ; }
\renewcommand*{\bibbfsasep} { ; }
\renewcommand*{\bibbstasep} { ; } %between second and third author sep
\renewcommand*{\jbbtesep} { ; } %between two editors sep
\renewcommand*{\jbbfsesep} { ; } %between first and second editor sep
\renewcommand*{\jbbstesep} { ; } %between second and third editor sep
\renewcommand*{\bibbtesep} { ; } %between two editors sep
\renewcommand*{\bibbfsesep} { ; } %between first and second editor sep
\renewcommand*{\bibbstesep} { ; } %between second and third editor sep
\AddTo\bibsgerman{\def\editorsname{(Hrsg.)}}
\AddTo\bibsgerman{\def\editorname{(Hrsg.)}}
%\jurabibsetup{super, citefull=first,ibidem}
%\jurabibsetup{ibidem}
%\jurabibsetup{authorformat=citationreversed}
%\jurabibsetup{authorformat=reducedifibidem}
\jurabibsetup{biblikecite}
%\jurabibsetup{bibformat=ibidem}
%\jurabibsetup{pages=always}
\jbfirstcitepageranges
\AddTo\bibsgerman{\def\herename{hier}}
\jbuseidemhrule

\jurabibsetup{
  authorformat={smallcaps,year,and,citationreversed},
  titleformat={colonsep,all,italic},
  commabeforerest,
  see,
  dotafter=bibentry,
  ibidem=strict,
  biblikecite
}

\renewcommand*{\bibbtasep}{ und } %
\renewcommand*{\bibbfsasep}{, }   %
\renewcommand*{\bibbstasep}{ und }
\renewcommand*{\jbtitlefont}{}
\renewcommand*{\bibtfont}{}
\renewcommand*{\bibbtfont}{}
\renewcommand*{\bibjtfont}{}
\renewcommand*{\bibapifont}{}
\renewcommand*{\jbshorttitlefont}{}



	%
	% CITATIONS
	%
\newcommand{\book}[2]{\footnote{\cite[Vgl.][#2]{#1}}}
\newcommand{\bookwf}[2]{\cite[Vgl.][#2]{#1}}
\newcommand{\bookdir}[2]{\footnote{\cite[][#2]{#1}}}
\newcommand{\inetwf}[1]{\cite[Vgl.][\citefield{url}{#1}]{#1}}
\newcommand{\inetwfdir}[1]{\cite[][\citefield{url}{#1}]{#1}}
\newcommand{\inet}[1]{\footnote{\inetwf{#1}}}
\newcommand{\inetdir}[1]{\footnote{\cite[][\citefield{url}{#1}]{#1}}}
\newcommand{\innerref}[1]{\footnote{Vgl. auch Kapitel \ref{#1} dieser Arbeit, S. \pageref{#1}}}
\newcommand{\vgl}[2]{\cite[Vgl.][#2]{#1}}
\newcommand{\citeauthoryear}[1]{\citeauthor{#1} (\citeyear{#1})}
\bibliographystyle{jurabib}

% setup of source code listings
\usepackage{listings}
%\usepackage{courier}
\usepackage{caption}
\lstset{
	basicstyle=\footnotesize\ttfamily,	% default font
	numbers=left,						% line numbers placement
	numberstyle=\tiny,					% line numbers style
	%stepnumber=2,						% line number padding
	numbersep=5pt,						% padding between line numbers and code
	tabsize=2,							% 
	extendedchars=true,         
	breaklines=true,						% line breaks 
	keywordstyle=\color{red},
	frame=b,
	stringstyle=\color{gray}\ttfamily,	% color of strings in code
	showspaces=false,					% visualize spaces
    showtabs=false,						% visualize tabs
    xleftmargin=17pt,
	framexleftmargin=17pt,
	framexrightmargin=5pt,
	framexbottommargin=4pt,
	showstringspaces=false				% visualize spaces in strings        
 }
 
 \lstloadlanguages{% Check docs for further languages ...
         C,
         C++,
         bash
 }

\setlength{\parindent}{0pt}
\setlength{\parskip}\medskipamount

\DeclareCaptionFont{white}{\color{white}}
\DeclareCaptionFormat{listing}{\colorbox{gray}{\parbox{\textwidth}{#1#2#3}}}
\captionsetup[lstlisting]{format=listing,labelfont=white,textfont=white}

% layout the box
%\DeclareCaptionFormat{listing}{\colorbox[rgb]{0.43, 0.35, 0.35 {\parbox{\textwidth}{\hspace{15pt}#1#2#3}}}

% layout the caption ontop of code
\captionsetup[lstlisting]{format=listing,labelfont=white,textfont=white, singlelinecheck=false, margin=0pt, font={bf,footnotesize}}

% Headings
\usepackage{fancyhdr}
\fancyhead[R]{\colorbox{blue!20}{ Oliver Erxleben}}
\fancyfoot{}

% Document begins now
\begin{document}

\author{%
	Martin Helmich \small(\href{mailto:martin.helmich@hs-osnabrueck.de}{martin.helmich@hs-osnabrueck.de})\\%
	Oliver Erxleben \small(\href{mailto:oliver.erxleben@hs-osnabrueck.de}{oliver.erxleben@hs-osnabrueck.de})\\ \\%
	%
	Hochschule Osnabr"uck \\%
	Ingenieurswissenschaften und Informatik \\%
	Informatik - Mobile und Verteilte Anwendungen }

\title{\includegraphics[scale=0.75,keepaspectratio]{img/hs_os.png}\linebreak \linebreak Entwicklung eines Treiber und einer Toolchain zur Administration eines Embedded Systems}

\maketitle
\thispagestyle{empty}
\tableofcontents

\listoffigures

\lstlistoflistings

\thispagestyle{empty}
\pagebreak
\thispagestyle{empty}
\begin{abstract}
\textbf{Zusammenfassung:}\\ 	
Die vorliegende Arbeit wurde mit LaTeX verfasst und ist eine gemeinsame Arbeit von Oliver Erxleben und Martin Helmich für das Modul \textit{Hardwarenahe System- und Treiberprogrammierung} aus dem Master-Studiengang \textit{Informatik - Verteilte und Mobile Anwendungen} im Wintersemester 2012/13 an der Hochschule Osnabrück / University of Applied Sciences. \\ 
\\
Das Thema der Arbeit lautet \textit{Entwicklung eines Treiber und einer Toolchain zur Administration eines Embedded Systems}. Im Kern beschreibt die Arbeit die Entwicklung eines Treibers für ein eingebettetes System unter Linux. Treiberprogrammierung unter anderen Systemen, wie zum Beispiel Microsoft Windows oder Apple Mac OS werden nicht betrachtet. Genauer wird zum einen ein Linux-Treiber für eine ZPU vorgestellt und zum anderen Bibliotheken und Wekrzeuge für die einfache Verwendung in Anwenderprogrammen zur Steuerung des ZPU entwickelt.\\
Im ersten Teil der Arbeit wird das Konzept der Hardware und das Konzept des Treibers und der Programmierbibliotheken vorgestellt. In den darauffolgenden Abschnitten werden detailliert die einzelnen Komponenten beschrieben. Anwenderprogramme zum Testen werden an geeigneten Stellen vorgestellt. \\ 
Das 
\\ Details über die Implementierung der Anwenderprogrammen wird aus Gründen der Lesbarkeit in den Anhang verschoben.\\ 
\\
% TODO Kernel Version ???
Sofern nicht anders angegeben, wird zum Kompilieren des Quellcodes der GNU-Compiler GCC unter Linux verwendet. 
\end{abstract}

\pagebreak
% set new page style

\pagestyle{fancy}
\setcounter{page}{1} 

\section{Einleitung}

% TODO: überarbeiten / erweitern
Täglich verwendet und verlässt sich unsere Gesellschaft auf computergestützte Anwendungen. Wir suchen, schreiben, drucken, kopieren, erstellen oder entfernen Daten. Nur allzu oft werden dabei Geräte verwendet, die Funktionen für den Anwender bereitstellen, oder aber das Funktionsspektrum des Comutersystems erweitern. Funktionen der Computerkomponenten werden durch Anwenderprogramme durchgeführt, die von einem Betriebssystem verwaltet werden. Das Betriebssystem verwaltet auch alle Komponenten des Computersystems. Diese Komponenten sind Software oder auch Hardware. Erst die Teamarbeit zwischen Hardware und Software ermöglicht für den Anwender die Ausführung komplexer Programme. \\
Das Betriebssystem steuert den Zugriff auf Hardware und benötigt Kenntnisse über die angeschlossenen Hardware-Komonenten. Das \textit{Wissen} über angeschlossene Hardwarekomponenten ist in sog. Gerätetreibern\footnote{Gerätetreiber:} hinterlegt und stellt einen Teil des Betriebssystemkerns dar, der für den Zugriff auf eine Hardware verantwortlich ist. Für jedes Gerät wird ein eigener Treiber benötigt.\\
Die Entwicklung eines Gerätetribers und zugehörige Softwarekomponenten stellen einen interessanten, wenn nicht sogar erstrebenswerten, Zweig der Programmierung von Computersystemen dar.

\section{Architektur}

\begin{figure}[!htb]
	\begin{center}
\begin{tikzpicture}
	\usetikzlibrary{positioning}
	\tikzstyle{node}=[anchor=mid,text centered,minimum height=2em];
	\tikzstyle{app}=[draw,fill=green!50,rectangle]
	\tikzstyle{lib}=[draw,fill=yellow!50,rectangle]
	\tikzstyle{dri}=[draw,fill=red!50,rectangle]
				
	\draw [lib] (6,0.5) rectangle +(2,0.75) node [node,midway] {libzpu};
	\draw [lib] (4cm-2pt,0.5) rectangle +(2,0.75) node [node,midway] {libintelhex};
				
	\draw [dri] (-0.5,-0.5) rectangle +(8.5,-0.75) node [node,midway] {modzpu};
	\draw [app] (6,2) rectangle +(2,0.75) node [node,midway] {zpuio};
	\draw [app] (4cm-2pt,2) rectangle +(2,0.75) node [node,midway] {zpuload};
					
	\draw [dashed] (0,0) -- (8,0);
	\draw [->] (7,0.5) -- (7,-0.5);
	\draw [->] (7,2) -- (7,1.25);
	\draw [->] (5,2) -- (5,1.25);
	\draw [->] (5.5,2) -- (6.5,1.25);
				
	\node [anchor=west] at (-0.5, 2.375) {Userspace-Programme};
	\node [anchor=west] at (-0.5, 0.875) {Bibliotheken};
\end{tikzpicture}
\end{center}
	\caption{Überblick über zu entwickelnde Architektur}
	\label{zpu_architecture}
\end{figure}


\begin{figure}[!htb]
	%include from externl
	\begin{tikzpicture}
    \tikzstyle{every node}=[font=\small]
	\usetikzlibrary{positioning}
	\tikzstyle{pci}=[draw,fill=green!50,rectangle]
	\tikzstyle{krn}=[draw,fill=yellow!50,rectangle]
				
	\draw [fill=green!20] (0,-0.5) rectangle +(8,2.5);
	\draw [fill=yellow!20] (0,2.5) rectangle +(8,2.5);
				
	%\draw [<->] (1.8,0.9) -- (5,0.9);
	%\draw [<->] (3.8,1.1) -- (5,1.1);
	\draw [<->,dotted] (6,1.5) -- (6,3);
			
	\draw [<->,dotted] (1,0.2) -- (1,-0.2) -- (6.1,-0.2) -- (6.1,0.5);
	\draw [<->,dotted] (3,0.2) -- (3,-0) -- (5.9,0) -- (5.9,0.5);
				
	\draw [<->,dotted] (3.2,4.3) -- (3.2,4.5) -- (5.9,4.5) -- (5.9,4);
	\draw [<->,dotted] (1.2,4.3) -- (1.2,4.7) -- (6.1,4.7) -- (6.1,4);
				
	%\draw [<->] (1.8,3.4) -- (5,3.4);
	%\draw [<->] (3.8,3.6) -- (5,3.6);
				
	\node [anchor=north east] at (8,-0.5) {PCI-Hardware};
	\node [anchor=south east] at (8, 5.0) {Kernel};
				
	\draw [pci] (0.2,0.2) rectangle +(1.6,1.6) node [align=left,midway] {Eingabe-\\FIFO\\(\texttt{stdin})};
	\draw [pci] (2.2,0.2) rectangle +(1.6,1.6) node [midway,align=left] {Ausgabe-\\FIFO\\(\texttt{stdout})};
					
	\draw [krn] (0.2,2.7) rectangle +(1.6,1.6) node [midway,align=left] {Eingabe-\\FIFO};
	\draw [krn] (2.2,2.7) rectangle +(1.6,1.6) node [midway,align=left] {Ausgabe-\\FIFO};
				
	\draw [->] (1,2.7) -- (1,1.8);
	\draw [<-] (3,2.7) -- (3,1.8);
				
	\draw [->] (1,5.5) -- (1,4.3);
	\draw [<-] (3,5.5) -- (3,4.3);
	\node [anchor=south] at (3,5.5) {\texttt{read()}};
	\node [anchor=south] at (1,5.5) {\texttt{write()}};
				
	\draw [pci] (5,0.5) rectangle +(2,1) node [align=left,midway] {Interrupt-\\Controller};
		\draw [krn] (5,3) rectangle +(2,1) node [midway] {IRQ-Handler};
\end{tikzpicture}
	\caption{ZPU-Ein und Ausgabe}
	\label{zpu_io}
\end{figure}


\section{Linux-Treiber-Modul: modzpu}

\section{ZPU Lib: libzpu}

\section{Intel Hex Lib: libcintelhex}

\pagebreak % content ends here

\fancyhead[R]{}

\thispagestyle{empty}

\renewcommand*{\biburlprefix}{(URL: }
\renewcommand*{\biburlsuffix}{)}

\pagebreak
\addcontentsline{toc}{section}{Literaturverzeichnis} % Eintrag ins Inhaltsverzeichnis
\bibliography{bib/bib}

\appendix

\end{document}
